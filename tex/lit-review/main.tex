\documentclass{article}

% \usepackage[margin=0.5in]{geometry}
\usepackage{arxiv}

\usepackage[utf8]{inputenc} % allow utf-8 input

\usepackage[T1]{fontenc}    % use 8-bit T1 fonts
\usepackage{hyperref}       % hyperlinks
\usepackage{url}            % simple URL typesetting
\usepackage{booktabs}       % professional-quality tables
\usepackage{amsfonts}       % blackboard math symbols
\usepackage{enumitem}
% \usepackage{nicefrac}       % compact symbols for 1/2, etc.
% \usepackage{microtype}      % microtypography
% \usepackage{lipsum}
\usepackage{multicol}
\usepackage{graphicx}
\usepackage{float}
\usepackage{subcaption}
\usepackage[ruled,vlined]{algorithm2e}


% \newcommand{\xsi}{x^{(i)}}
% \newcommand{\ysi}{y^{(i)}} 

\title{
    Suggesting arXiv Papers by Semantic Relation \\
    \vspace{0.5em}
    \small{cs224u Project - Literature Review}
}
\author{
Junshen Kevin Chen \\
Stanford University\\
\texttt{jkc1@stanford.edu} 
}

\begin{document}
\maketitle

% \begin{abstract}
% lorem ipsum
% \end{abstract}


\begin{multicols*}{2}

% ============================== begin content ==============================
\vspace{-15px}
\section*{Overview}

For this project, I propose to design a system that evaluates the level of relation between two academic papers, and suggest possible related work given a text query.

\paragraph{Motivation} 
Three main factors contribute to motivating this project. 
First, the current state of academic research tools such as Google Scholar functioning similar to a traditional search engine, which primarily leverages keyword matching and ranking by popularity (citation counts), while having being limitedly influenced by content semantics. 
Second, large data dumps of academic papers such as arXiv is available to extract relationships between academic work in the form of a citation graph.
Finally, the research in transfomer models such as BERT [todo] provides powerful ways to encode semantics, and trained models are readily available to be fine-tuned for this specific task. 

\paragraph{Problem}
Tentatively, I formulate this project into several following problems, with each latter depending on the former:
\begin{enumerate}[leftmargin=*]
    \item Given two articles, produce a probability of citation, or "how likely does one cite another"
    \item Given two articles, predict a "citation distance", or how many edges it needs to traverse from one to another
    \item Given a query text, possibly the abstract of the working paper, recommend related articles 
\end{enumerate}

In this document, I select related works that provide context to the proposed project, in hopes that the outcome of this project will be able to perform exactly this selection task.

\section{arXiv Dataset and Citation Graph}




\newpage

\bibliographystyle{unsrt}
\bibliography{reference}


\end{multicols*}

\end{document}


