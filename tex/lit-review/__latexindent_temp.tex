\documentclass{article}

% \usepackage[margin=0.5in]{geometry}
\usepackage{arxiv}

\usepackage[utf8]{inputenc} % allow utf-8 input

\usepackage[T1]{fontenc}    % use 8-bit T1 fonts
\usepackage{hyperref}       % hyperlinks
\usepackage{url}            % simple URL typesetting
\usepackage{booktabs}       % professional-quality tables
\usepackage{amsfonts}       % blackboard math symbols
\usepackage{enumitem}
% \usepackage{nicefrac}       % compact symbols for 1/2, etc.
% \usepackage{microtype}      % microtypography
% \usepackage{lipsum}
\usepackage{multicol}
\usepackage{graphicx}
\usepackage{float}
\usepackage{subcaption}
\usepackage[ruled,vlined]{algorithm2e}


% \newcommand{\xsi}{x^{(i)}}
% \newcommand{\ysi}{y^{(i)}} 

\title{Finding Related arXiv Papers by Abstract}

% \vspace{-30pt}
\author{
Junshen Kevin Chen \\
Stanford University\\
\texttt{jkc1@stanford.edu} 
}

\begin{document}
\maketitle

% \begin{abstract}
% This is the milestone report for our final project. In the work so far, we have formalized data collection methodology and collected a small sample of preliminary data. This report also discusses some of the key strategies in organizing and processing data for our training work. Then, a preliminary K-means and a K-Medoids model is introduced as a baseline. Finally, this report discusses some of the potential future plans.
% \end{abstract}


\begin{multicols*}{2}

% ============================== begin content ==============================
\vspace{-15px}
\section{Introduction}
as


\newpage

\bibliographystyle{unsrt}
\bibliography{reference}


\end{multicols*}

\end{document}


